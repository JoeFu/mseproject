%%%%%%weekly meeting template, prepared by Michael Sheng.  09/03/2007
\documentclass[11pt, a4paper]{article}
\usepackage{times}
\usepackage{ifthen}
\usepackage{amsmath}
\usepackage{amssymb}
\usepackage{graphicx}
\usepackage{setspace}

%%% page parameters
\oddsidemargin -0.5 cm
\evensidemargin -0.5 cm
\textwidth 15 cm
\topmargin -1.2 cm
\textheight 22 cm

\renewcommand{\baselinestretch}{1.4}\normalsize
\setlength{\parskip}{0pt}
				

\begin{document}


\vspace*{15pt}

\begin{center}
\huge Meeting Minutes




\end{center}

\section{Date and Time}
 Wednesday 10th May 2017 2:00pm-3:00pm IW 4.36

\section{Apologies}
None

\section{Present}
\begin{itemize}
	\item Nickolas Falkner
	\item Christoph Treude 
	\item Minh Tam Phan
	\item Zeqi Fu
	\item Zeyu Lin
\end{itemize}

\section{Decisions that were made and issues clarified}
\begin{itemize}
	\item For CSV file, line 1 contains titles (names of columns), then the data. Incorporate more information in the file name.
	\item The dashboard should tell which course it is.
	\item Remove the comma of number (e.g 2,500) in charts. The solution is amount of activities (in thousands)
	\item Preference for the x-label: Jul 12.
	\item Provide choices (alphabetical order, descending order, ascending order) for charts.
	\item We need to provide a threshold function which also deal with "0" case.
	\item For the x-label issue, remove the labels until they can be displayed correctly. Shrink the x-label user0019 to u0019.
	\item For charts related with "Event name", provide one filter to set threshold and then another to auto-complete the event name you type.
	\item For the questionnaire, describe the Moodle and WebSubmission data in the first paragraph. For question1: We want to identify some commonly asked questions so that we can calculate the answers to these questions in our application. Based on schema of Moodle and WebSubmission, how would 
you relate the data to answer the questions? We don't need question 2.
	\item For the report, give more space for the chart, the chart
should be as big as possible, the report summary goes underneath.
The chart, the source data, when the report generated, any identifying aspects of the report itself are important.
	\item The client will provide a list of people for the survey.

\end{itemize}

\section{Action items for specific people (due date)}
\begin{itemize}
	\item Minh Tam Phan: Implement simple query(May 16)
	\item Zeqi Fu: Modify charts, provide more functions.(May 16)
	\item Zeyu Lin: Modify charts, provide more functions.(May 16)

\end{itemize}



\end{document}